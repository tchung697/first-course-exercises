\documentclass[10pt]{article}
\title{Solutions to Exercises of Chapter 6}
\author{Ting-Chih Hung \\ \href{mailto:nuit0815@gmail.com}{\texttt{nuit0815@gmail.com}}}
\date{\today}

\usepackage{settings}

\begin{document}
\maketitle

\section*{Problem 6.7 (Lin's estimator for covariate adjustment)}

Consider two \gls{ols} problems:
\begin{align*}
  \min_{\gamma_1, \beta_1} 
  \sum_{i=1}^n Z_i \bp{Y_i - \gamma_1 - \beta_1^\intercal X_i}^2
  \quad \text{and} \quad
  \min_{\gamma_0, \beta_0} 
  \sum_{i=1}^n (1 - Z_i) \bp{Y_i - \gamma_0 - \beta_0^\intercal X_i}^2.
\end{align*}
The solutions are $\hat{\gamma}_1, \hat{\beta}_1$ and $\hat{\gamma}_0, \hat{\beta}_0$.
Define the estimator
\begin{align}
  \hat{\tau}(\hat{\beta}_1, \hat{\beta}_0) 
  &= \frac{1}{n_1} \sum_{i=1}^n Z_i \bp{Y_i - \hat{\beta}_1^\intercal X_i}
  - \frac{1}{n_0} \sum_{i=1}^n (1 - Z_i) \bp{Y_i - \hat{\beta}_0^\intercal X_i} \notag \\
  &= \hat{\gamma}_1 - \hat{\gamma}_0.
  \tag{6.4}
  \label{eq:6.4}
\end{align}

We are asked to show the Proposition 6.2:
\begin{quote}
  The estimator $\hat{\tau}(\hat{\beta}_1, \hat{\beta}_0)$ 
  in \cref{eq:6.4} equals the coefficient of $Z_i$ in the \gls{ols}
  fit of $Y_i$ on $\bp{1, Z_i, X_i, Z_i \times X_i}$,
  which is Lin's estimator, $\hat{\tau}_{\text{L}}$.
\end{quote}

Let's write the \gls{ols} problem of regressing $Y_i$ on $\bp{1, Z_i, X_i, Z_i \times X_i}$:
\begin{align*}
  \min_{\alpha_0, \alpha_1, \alpha_2, \alpha_3}
  \sum_{i=1}^n \bp{Y_i - \alpha_0 - \alpha_1 Z_i - 
    \alpha_2^\intercal X_i - \alpha_3^\intercal (Z_i X_i)}^2.
\end{align*}
It is equivalent to the following problem:
\begin{align*}
  \min_{\alpha_0, \alpha_1, \alpha_2, \alpha_3}
  \sum_{i=1}^n Z_i \bp{Y_i - (\alpha_0 + \alpha_1) - (\alpha_2 + \alpha_3)^\intercal X_i}^2
  + \sum_{i=1}^n (1 - Z_i) \bp{Y_i - \alpha_0 - \alpha_2^\intercal X_i}^2.
\end{align*}
Let $\gamma_1 = \alpha_0 + \alpha_1$, $\beta_1 = \alpha_2 + \alpha_3$, $\gamma_0 = \alpha_0$, and $\beta_0 = \alpha_2$.
Then the above problem becomes
\begin{align*}
  \min_{\gamma_1, \beta_1, \gamma_0, \beta_0}
  \sum_{i=1}^n Z_i \bp{Y_i - \gamma_1 - \beta_1^\intercal X_i}^2
  + \sum_{i=1}^n (1 - Z_i) \bp{Y_i - \gamma_0 - \beta_0^\intercal X_i}^2.
\end{align*}
This is exactly a reparametrization of the two separate \gls{ols} problems we started with.
Therefore, the solutions are related by
\begin{align*}
  \hat{\gamma}_1 &= \hat{\alpha}_0 + \hat{\alpha}_1, \\
  \hat{\beta}_1 &= \hat{\alpha}_2 + \hat{\alpha}_3, \\
  \hat{\gamma}_0 &= \hat{\alpha}_0, \\
  \hat{\beta}_0 &= \hat{\alpha}_2,
\end{align*}
and we have
\begin{align*}
  \hat{\tau}(\hat{\beta}_1, \hat{\beta}_0)
  = \hat{\gamma}_1 - \hat{\gamma}_0 
  = (\hat{\alpha}_0 + \hat{\alpha}_1) - \hat{\alpha}_0 
  = \hat{\alpha}_1 
  = \hat{\tau}_{\text{L}}.
\end{align*}

\section*{Problem 6.8 (Predictive and projective estimators)}

We can predict $Y_i(z)$ based on $X_i$ using the data 
from the treatment group ($Z_i =1$) and control group ($Z_i = 0$) separately,
\begin{align*}
  \hat{\mu}_1(X_i) &= \hat{\gamma}_1 + \hat{\beta}_1^\intercal X_i, \\
  \hat{\mu}_0(X_i) &= \hat{\gamma}_0 + \hat{\beta}_0^\intercal X_i,
\end{align*}
where $(\hat{\gamma}_1, \hat{\beta}_1)$ and $(\hat{\gamma}_0, \hat{\beta}_0)$
are the solutions to the two \gls{ols} problems in Problem 6.7.

If we use the predictions to impute the missing potential outcomes,
then we can estimate the \gls{ate} by
\begin{align*}
  \hat{\tau}_{\text{pred}}
  &= \frac{1}{n} 
  \bc{\sum_{Z_i = 1} Y_i + \sum_{Z_i = 0} \hat{\mu}_1(X_i)
  - \sum_{Z_i = 1} \hat{\mu}_0(X_i) - \sum_{Z_i = 0} Y_i}, \tag{6.7}
  \label{eq:6.7}
\end{align*}
which is called the \emph{predictive estimator}.

Instead of predicting the missing potential outcomes,
if we predict all potential outcomes even if they are observed,
then we have the \emph{projective estimator}:
\begin{align*}
  \hat{\tau}_{\text{proj}}
  &= \frac{1}{n} \sum_{i=1}^n \bc{\hat{\mu}_1(X_i) - \hat{\mu}_0(X_i)}. \tag{6.9}
\end{align*}

We are asked to show that 
$\hat{\tau}_{\text{pred}}$ equals Lin's estimator $\hat{\tau}_{\text{L}}$.
Furthermore, $\hat{\tau}_{\text{proj}}$ also agrees with $\hat{\tau}_{\text{L}}$.

We first show $\hat{\tau}_{\text{proj}} = \hat{\tau}_{\text{L}}$.
By substituting the definitions of $\hat{\mu}_1(X_i)$ and $\hat{\mu}_0(X_i)$,
\begin{align*}
  \hat{\tau}_{\text{proj}}
  &= \frac{1}{n} \sum_{i=1}^n \bc{(\hat{\gamma}_1 - \hat{\gamma}_0) + (\hat{\beta}_1 - \hat{\beta}_0)^\intercal X_i} \\
  &= (\hat{\gamma}_1 - \hat{\gamma}_0) + (\hat{\beta}_1 - \hat{\beta}_0)^\intercal \bar{X} \\
  &= \hat{\tau}_{\text{L}} + (\hat{\beta}_1 - \hat{\beta}_0)^\intercal \bar{X} \\
  &= \hat{\tau}_{\text{L}},
\end{align*}
where the last equality holds because of the centering of $X_i$ (i.e., $\bar{X} = 0$).

By normal equations of the two \gls{ols} problems,
we have
\begin{align*}
  \sum_{Z_i = z} Y_i 
  &= \sum_{Z_i = z} \bp{\hat{\gamma}_z + \hat{\beta}_z^\intercal X_i}
  = \sum_{Z_i = z} \hat{\mu}_z(X_i).
\end{align*}
Plug in the above equation into \cref{eq:6.7}, we have
\begin{align*}
  \hat{\tau}_{\text{pred}}
  &= \frac{1}{n} 
  \bc{\sum_{Z_i = 1} \hat{\mu}_1(X_i) + \sum_{Z_i = 0} \hat{\mu}_1(X_i)
  - \sum_{Z_i = 1} \hat{\mu}_0(X_i) - \sum_{Z_i = 0} \hat{\mu}_0(X_i)} \\
  &= \frac{1}{n} \sum_{i=1}^n \bc{\hat{\mu}_1(X_i) - \hat{\mu}_0(X_i)} \\
  &= \hat{\tau}_{\text{proj}} \\
  &= \hat{\tau}_{\text{L}}.
\end{align*}

\section*{Problem 6.11 (Regression adjustment / post-stratification of CRE)}

Suppose that we have discrete covariates $C_i \in \{1, 2, \ldots, K\}$.
We create $K - 1$ centered dummy variables
\begin{align*}
  X_{i} = \bc{\mathbf{1}\{C_i = 1\} - \pi_{[1]}, \ldots, \mathbf{1}\{C_i = K - 1\} - \pi_{[K - 1]}},
\end{align*}
where $\pi_{[k]} = P(C_i = k)$ for $k = 1, \ldots, K$.
We are asked to show that Lin's estimator $\hat{\tau}_{\text{L}}$
based on $X_i$ equals the post-stratified estimator $\hat{\tau}_{\text{PS}}$:
\begin{align*}
  \hat{\tau}_{\text{PS}}
  &= \sum_{k=1}^K \hat{\pi}_{[k]} \hat{\tau}_{[k]}.
\end{align*}

Define $\beta_z = (\beta_{z, 1}, \ldots, \beta_{z, K - 1})^\intercal$ for $z \in \{0, 1\}$.
Let the predicted potential outcomes be
\begin{align*}
  \hat{\mu}_1(X_i) &= \hat{\gamma}_1 + \hat{\beta}_1^\intercal X_i, \\
  \hat{\mu}_0(X_i) &= \hat{\gamma}_0 + \hat{\beta}_0^\intercal X_i,
\end{align*}
where $(\hat{\gamma}_1, \hat{\beta}_1)$ and $(\hat{\gamma}_0, \hat{\beta}_0)$
are the regression coefficients from the regression of $Y_i$ on $(1, X_i)$
in the treatment group and control group, respectively.
Note that both regressions are saturated models within each stratum,
so the predicted potential outcomes equal the stratum means:
\begin{align*}
  \hat{\mu}_1(X_i) 
  &= \hat{\bar{Y}}_{[k]}(1) \quad \text{if } C_i = k, \\
  \hat{\mu}_0(X_i)
  &= \hat{\bar{Y}}_{[k]}(0) \quad \text{if } C_i = k.
\end{align*}
Thus,
\begin{align*}
  \hat{\tau}_{\text{L}}
  &= \frac{1}{n} \sum_{i=1}^n \bc{\hat{\mu}_1(X_i) - \hat{\mu}_0(X_i)} \\
  &= \frac{1}{n} \sum_{k=1}^K \sum_{C_i = k} 
  \bc{\hat{\bar{Y}}_{[k]}(1) - \hat{\bar{Y}}_{[k]}(0)} \\
  &= \sum_{k=1}^K \frac{n_{[k]}}{n} \hat{\tau}_{[k]} \\
  &= \sum_{k=1}^K \hat{\pi}_{[k]} \hat{\tau}_{[k]} \\
  &= \hat{\tau}_{\text{PS}}.
\end{align*}

\section*{Problem 6.12 (More on the difference-in-difference estimator in the CRE)}

\printglossaries
\end{document}
